\documentclass{article}
%%%%%%%%%%%%%%%%%%%%%%%%%%%%%%%%%%%%%%%%%%%%%%%%%%%%%%%%%%%%%
% Lecture Specific Information to Fill Out
%%%%%%%%%%%%%%%%%%%%%%%%%%%%%%%%%%%%%%%%%%%%%%%%%%%%%%%%%%%%%
\newcommand{\LectureTitle}{Lectures Week \#5 Notes}
%\newcommand{\LectureDate}{\today}
\newcommand{\LectureDate}{August\ 27,\ 2013}
\newcommand{\LectureClassName}{COMP\ 4121}
\newcommand{\LatexerName}{Dr\ Aleks\ Ignjatovic}
%%%%%%%%%%%%%%%%%%%%%%%%%%%%%%%%%%%%%%%%%%%%%%%%%%%%%%%%%%%%%
\newcommand{\Lagr}{\mathcal{L}}

% Change "article" to "report" to get rid of page number on title page
\usepackage{amsmath,amsfonts,amsthm,amssymb}
\usepackage{setspace}
\usepackage{Tabbing}
\usepackage{fancyhdr}


\usepackage{lastpage}
\usepackage{extramarks}
\usepackage{chngpage}
\usepackage{soul,color}
\usepackage{graphicx,float,wrapfig}
\usepackage{afterpage}
\usepackage{abstract}

\usepackage[parfill]{parskip}
% recommended from http://poisson.phc.unipi.it/~maggiolo/index.php/2008/12/latex-class-for-lecture-notes/
\usepackage[pdf]{pstricks}
\usepackage{pst-plot}
\usepackage{hyperref}
\usepackage{mathtools}
\usepackage{booktabs}
\usepackage{multirow}
\usepackage{fancyhdr}
\usepackage{mparhack}
\usepackage{tikz}
\usepackage{mathdots}
\usepackage{xfrac}
\usepackage{faktor}
\usepackage{cancel}

\usepackage{parcolumns} 
\usepackage{mdframed}

% In case you need to adjust margins:
\topmargin=-0.45in
\evensidemargin=0in
\oddsidemargin=0in
\textwidth=6.5in
\textheight=9.0in
\headsep=0.25in

% Setup the header and footer
\pagestyle{fancy}
\lhead{\LatexerName}
\chead{\LectureClassName: \LectureTitle}
\rhead{\LectureDate}
\lfoot{\lastxmark}
\cfoot{}
\rfoot{Page\ \thepage\ of\ \pageref{LastPage}}
\renewcommand\headrulewidth{0.4pt}
\renewcommand\footrulewidth{0.4pt}

%%%%%%%%%%%%%%%%%%%%%%%%%%%%%%%%%%%%%%%%%%%%%%%%%%%%%%%%%%%%%
% Some tools
\newcommand{\enterTopicHeader}[1]{\nobreak\extramarks{#1}{#1 continued on next page\ldots}\nobreak
                                    \nobreak\extramarks{#1 (continued)}{#1 continued on next page\ldots}\nobreak}
\newcommand{\exitTopicHeader}[1]{\nobreak\extramarks{#1 (continued)}{#1 continued on next page\ldots}\nobreak
                                   \nobreak\extramarks{#1}{}\nobreak}

\newlength{\labelLength}
\newcommand{\labelAnswer}[2]
  {\settowidth{\labelLength}{#1}
   \addtolength{\labelLength}{0.25in}
   \changetext{}{-\labelLength}{}{}{}
   \noindent\fbox{\begin{minipage}[c]{\columnwidth}#2\end{minipage}}
   \marginpar{\fbox{#1}}

   % We put the blank space above in order to make sure this
   % \marginpar gets correctly placed.
   \changetext{}{+\labelLength}{}{}{}}

\setcounter{secnumdepth}{0}
\newcommand{\TopicName}{}
\newcounter{TopicCounter}
\newenvironment{Topic}[1][Problem \arabic{TopicCounter}]
  {\stepcounter{TopicCounter}
   \renewcommand{\TopicName}{#1}
   \section{\TopicName}
   \enterTopicHeader{\TopicName}}
  {\exitTopicHeader{\TopicName}}
  
\setcounter{secnumdepth}{0}
\newcommand{\ExampleSectionName}{}
\newcounter{ExampleSectionCounter}[TopicCounter]
\newenvironment{ExampleSection}[1][Example \arabic{ExampleSectionCounter}]
  {\stepcounter{ExampleSectionCounter}
   \renewcommand{\ExampleSectionName}{#1}
   \section{\ExampleSectionName}
   \enterTopicHeader{\ExampleSectionName}}
  {\exitTopicHeader{\ExampleSectionName}}

\setcounter{secnumdepth}{0}
\newcounter{ExampleBoxCounter}[TopicCounter]
\newcommand{\examplebox}[1]
  {
  % We put this space here to make sure we're disconnected from the previous
   % passage
   \stepcounter{ExampleBoxCounter}
   \noindent\fbox{\begin{minipage}[c]{\columnwidth}#1\end{minipage}}\enterTopicHeader{\ExampleSectionName}\exitTopicHeader{\ExampleSectionName}\marginpar{\fbox{\#\arabic{ExampleBoxCounter}}}
   % We put the blank space above in order to make sure this
   % \marginpar gets correctly placed.
   \vskip10pt
   }

\renewcommand{\contentsname}{{\normalsize Topics Covered}}
\renewcommand{\abstractname}{\LectureTitle\ Summary}
\renewcommand{\absnamepos}{flushleft}

%%%%%%%%%%%%%%%%%%%%%%%%%%%%%%%%%%%%%%%%%%%%%%%%%%%%%%%%%%%%%

\begin{document}
\begin{spacing}{1.1}
\newpage

%\begin{abstract}
%Prepared by Laurence Davies
%\end{abstract}

\tableofcontents
\addtocontents{toc}{~\hfill\textbf{Page}\par}
\vskip10pt
\hrule
\vskip10pt
\newpage

% When topics are long, it may be desirable to put a \newpage or a
% \clearpage before each Topic environment
%\newpage
\begin{Topic}[\Roman{TopicCounter} Random Variables]

\begin{itemize}
\item $X_{1},...,X_{n}$  are independent if $\displaystyle P(X_{1}\in A_{1} ^ ... ^ X_{n}\in A_{n})=\prod_{i=1}^{n}P(X_{i}\in A_{i})$
\item IID - independent identically distributed variables $X_{1},...,X_{n}$ IID $\rightarrow$ $\langle X_{1},...,X_{n} \rangle$ is a \textit{random sample of size n}. 
\item A function of an RV is also an RV.
\item Sample mean $\displaystyle \bar{X}=\frac{X_{1}+...+X_{n}}{n}$
%\item  
%  \begin{itemize}
%    \item (Rotations, reflections, glide reflections, similarities) 
%  \end{itemize}
\end{itemize}

Assime $X_{1},...,X_{n}$ are IID, with 
\begin{align*}
E(X_{i})&=\mu \\
V(X_{i})&=E(X-E(X))^{2}=\sigma^{2}
\end{align*}

\textbf{Then:} 
\begin{align*}
E(\bar{X})&=\mu \\
V(\bar{X})&=\frac{\sigma}{n}
\end{align*}

Evaluating with the sample mean gives:

\begin{align*}
E(\bar{X}) &=\frac{n E(X_{1})}{n} \\&= E(X_{1}) \\
\end{align*}

\begin{align*}
E(\bar{X}-\mu)^{2} &=E(\frac{\sum_{i=1}{n}(X_{i}-\mu)}{n})^{2}\\
&=\frac{1}{n^{2}}(\sum_{i=1}{n}(X_{i}-\mu)^{2}+\sum_{i\ne j}{n}(X_{i}-\mu)(X_{j}-\mu))\\
&= \frac{1}{n^{2}}\cdot n\sigma^{2}\\
&= \frac{\sigma^{2}}{n}
\end{align*}

So $\bar{X}$ is an unbiased estimator of the true value (a.k.a. the population mean). What is an unbiased estimator of the variance $\displaystyle \sum_{i} (X_{i}-\bar{X})^{2}$?

\begin{align*}
E(\sum_{i} (X_{i}-\bar{X})^{2})
&= E(\sum_{i} (X_{i}^{2}-2X_{i}\bar{X} + \bar{X}^{2}))\\
&= E(\sum_{i} X_{i}^{2}-2(\sum_{i} X_{i})\bar{X} + n\bar{X}^{2})\\
&= E(nX^{2}-2n\bar{X}^{2} + n\bar{X}^{2})\\
&= nE(nX^{2})-nE(\bar{X}^{2})
\end{align*}

What is $E(X^{2})$?

\begin{align*}
\sigma^{2} &= E(X-\mu)^{2}\\
 &= E(X^{2}-2\mu X + \mu^{2})\\
 &= E(X^{2})-2\mu^{2} + \mu^{2})\\
 &= E(X^{2})-\mu^{2})\\
\end{align*}

So 

\begin{align}
E(X^{2})&=\sigma^{2}+\mu^{2}
\end{align}

So we get

\begin{align*}
E(\sum_{i} (X_{i}-\bar{X})^{2})
 &= n\cdot (\sigma^{2} + \mu^{2}) - n\cdot (\frac{\sigma^{2}}{n}- \mu^{2})\\
 &= n\sigma^{2} + n\mu^{2} - \sigma^{2}- n\mu^{2}\\
 &= n\sigma^{2} + \sigma^{2}\\
 &= (n-1)\sigma^{2}\\
\end{align*}

So an unbiased variance estimator is

\begin{align}
\frac{1}{n-1}\sum_{i}(X_{i}-\bar{X})^{2}
\end{align}

\end{Topic}

\newpage
\begin{Topic}[\Roman{TopicCounter} Likelihood]
Assume you have a coin with $P(H)=p$ which is unknown. You toss it 16 times and get\\

\begin{align*}
HHTHTHHHTTHHTHHT
\end{align*}

For what value of $p$ is such an outcome \textit{most likely}? We have 10 heads so intuitively $p$ should be $\frac{10}{16}$. To show this, start with\\

\begin{align*}
P(p)&=p^{10}(1-p)^{6}\\
\frac{\partial P}{\partial p} &= 10p^{9}(1-p)^{6} + p^{10}\cdot 6(1-p)^{5}(-1)\\
&= p^{9}(1-p)^{5}(10(1-p)-6p)\\
&= p^{9}(1-p)^{5}(10-10p-6p)\\
\end{align*}

So $P(p)$ has a maximum value
\begin{align*}
16p&=10\\
p&=\frac{10}{16}
\end{align*}

The likelihood function is \textbf{not} a probability function on the space of parameters. It is equal to:
\begin{enumerate}
\item In the discrete case: The probability to have such an outcome given the unknown parameters.
\item In the continuous case: The probability density function for such an outcome.
\end{enumerate}

\begin{ExampleSection}
Assume you have $n$\textbf{unbiased} sensors with an unknown but equal standard deviation.
You have $n$ readings $X_{1},...,X_{n}$ of the same quantity. How would you estimate the true value $\mu$ and $\sigma$?

\begin{align*}
\Lagr_{n}(\mu,\sigma)&=\prod_{i=1}^{n}\frac{1}{\sqrt{2\pi\sigma^{2}}}e^{-\frac{(X_{i}-\mu)^{2}}{2\sigma^{2}}}\\
\frac{\partial\Lagr_{n}}{\partial\mu}&=\frac{\partial}{\partial\mu}(\frac{1}{(2\pi\sigma^{2})^{\frac{n}{2}}}e^{-\frac{\sum_{i}(X_{i}-\mu)^{2}}{2\sigma^{2}}})\\
&=\frac{1}{(2\pi\sigma^{2})^{\frac{n}{2}}}e^{-\frac{\sum_{i}(X_{i}-\mu)^{2}}{2\sigma^{2}}}\cdot\frac{-2}{2\sigma^{2}}\sum_{i}(X_{i}-\mu) 
  \text{   because } \mu=\frac{\sum_{i}X_{i}}{n} \\
&= -n(2\pi)^{\frac{-n}{2}}\sigma^{-n-1}e^{-\frac{\sum_{i}(X_{i}-\mu)^{2}}{2\sigma^{2}}}+
   (2\pi)^{\frac{-n}{2}}\sigma^{-n}e^{-\frac{\sum_{i}(X_{i}-\mu)^{2}}{2\sigma^{2}}}
   (-2) \frac{-\sum_{i}(X_{i}-\mu)^{2}}{2\sigma^{3}}\\
\end{align*}

\begin{align*}
\frac{\partial\Lagr_{n}}{\partial\mu}=0 
&\leftrightarrow
-n+(\sigma^{-1})^{2}=0\\
&\leftrightarrow
\sum_{i}(X_{i}-\m)^{2}=0\\
\sigma^{2}&=\frac{\sum_{i}(X_{i}-\m)^{2}}{n}
\end{align*}

We found that $\mu=\bar{X}$, so
\begin{align*}
\sigma^{2}&=\frac{\sum_{i}(X_{i}-\m)^{2}}{n}
\end{align*}
is a biased estimator.

\end{ExampleSection}

\begin{ExampleSection}
Given $n$ sensors with known variances $\sigma_{1}^{2},...,\sigma_{n}^{2}$ and $n$ readings $X_{1},...,X_{n}$. What is the maximum likelihood estimation of $\mu$?

\begin{align*}
\Lagr_{n}(\vec{X},\mu,\sigma)&=\prod_{i=1}^{n}\frac{1}{(2\pi)^{\frac{n}{2}}\sigma_{i}}e^{-\frac{(X_{i}-\mu)^{2}}{2\sigma^{2}}}\\
&=(2\pi)^{-\frac{n}{2}}\prod_{i=1}^{n}\sigma_{i}^{-1}\cdot e^{-\sum_{i}\frac{(X_{i}-\mu)^{2}}{2\sigma^{2}}}\\
\frac{\partial\Lagr_{n}}{\partial\mu }
&=(2\pi)^{-\frac{n}{2}}\prod_{i=1}^{n}\sigma_{i}^{-1}\cdot e^{-\sum_{i}\frac{(X_{i}-\mu)^{2}}{2\sigma^{2}}} - 2\sum_{i}\frac{X_{i}-\mu}{2\sigma_{i}}\\
\end{align*}

So
\begin{align*}
\frac{\partial\Lagr_{n}}{\partial\mu} = 0
&\leftrightarrow
\sum_{i}\frac{X_{i}}{\sigma_{i}}
=\sum_{i}\frac{1}{\sigma_{i}}\mu\\
&\leftrightarrow
\mu=
\frac{\sum_{i}\frac{1}{\sigma_{i}}X_{i}}{\sum_{k}\frac{1}{\sigma_{k}}}\\
\end{align*}


\end{ExampleSection}

\end{Topic}

\begin{Topic}[\Roman{TopicCounter} Maximum Likelihood Estimates]
Is the ML estrimate always a "good" one?

\begin{ExampleSection}
You have a box with a certain number of balls, numbered consecutively $1,2,3,4,...$.
You pick one at random, see its number and have to estimate the total number of balls in the box.\\
\textbf{Maximum Likelihood Estimate}: If you picked a ball with a number $k$, such a ball is
most likely if there are $k$ balls in the box, because then the probability of picking this ball is
$\frac{1}{k}$. The expected value of your estimate is

\begin{align*}
\frac{1+2+...+n}{n} &= \frac{n(n-1)}{2n}\\
&=\frac{n-1}{2}
\end{align*}

So your ML is heavily biased! On the other hand, if your estimate is $2X-1$ then

\begin{align*}
E(2X-1) &= \sum_{i}\frac{2\cdot i-1}{n}\\
&= \frac{2n(n+1)-n}{n}\\
&= n
\end{align*}

hence this estimate is unbiased.\\

Then to evaluate

\begin{align*}
T_{i} &= C-\frac{1}{m_{i}}\sum_{i\rightarrow p}(E_{ip}-\sqrt{p})^{2}
\end{align*}

We maximise

\begin{align*}
\tau &= \sum_{i}m_{i}T_{i}^{2} \\
     &= \sum_{i}m_{i}( C-\frac{1}{m_{i}}\sum_{i\rightarrow p}(E_{ip}-\sqrt{p})^{2} )^{2} \\
\frac{\partial\tau}{\partial\sqrt{k}} 
     &= 4\sum_{i\rightarrow k}m_{i}( C-\frac{1}{m_{i}}\sum_{i\rightarrow p}(E_{ip}-\sqrt{p})^{2} )^{2} \cdot\frac{E_{ik}-\sqrt{k}}{m_{i}}\\
     &= 4\sum_{i\rightarrow k}T_{i}( E_{ik}-\sqrt{k}) \\
     &= 4\sum_{i\rightarrow k}T_{i}E_{ik}-(\sum_{i\rightarrow k}T_{i})\sqrt{k} \\
\end{align*}

Equating the derivative to zero

\begin{align*}
\frac{\partial\tau}{\partial\sqrt{k}} = 0
     &\leftrightarrow
\sqrt{k}=\frac{\sum_{i\rightarrow k}T_{i}E_{ik}}{\sum_{i\rightarrow k}T_{i}}
\end{align*}

\end{ExampleSection}

\begin{ExampleSection}
What is the likelihood of obtaining readings $E_{ip}$ if all sensors have variance $\sigma^{2}$?

\begin{align*}
\Lagr_{n}(E^{i},\vec{r},\sigma)&=\prod_{i\rightarrow k}\frac{1}{\sqrt{2\pi\sigma}}e^{-\frac{(E_{ik}-\sqrt{k})^{2}}{2\sigma}}\\
ln(\Lagr_{n}(E^{i},\vec{r},\sigma))&=\sum_{i\rightarrow k}-\frac{1}{2}ln(2\pi\sigma)-\frac{(E_{ik}-\sqrt{k})^{2}}{2\sigma}\\
&=-\frac{1}{2}ln(2\pi\sigma)\cdot m_{i}-\frac{\sum_{i\rightarrow k}(E_{ik}-\sqrt{k})^{2}}{2\sigma}\\
\frac{2\sigma^{2}}{m_{i}}ln(\Lagr_{n}(E^{i},\vec{r},\sigma))&=-\sigma^{2}ln(2\pi\sigma)-\sum_{i\rightarrow k}(E_{ik}-\sqrt{k})^{2}\\
\frac{2\sigma^{2}}{m_{i}}ln(\Lagr_{n}(E^{i},\vec{r},\sigma))&= [ -\sigma^{2}ln(2\pi\sigma)-C ] + [ C -\sum_{i\rightarrow k}(E_{ik}-\sqrt{k})^{2} ]\\
\end{align*}

where $C$ is chosen such that $[ -\sigma^{2}ln(2\pi\sigma)-C ]=0$. Let $T_{i}= C -\sum_{i\rightarrow k}(E_{ik}-\sqrt{k})^{2}$, and so

\begin{align*}
(\frac{2\sigma^{2}}{m_{i}}ln(\Lagr_{n}(E^{i},\vec{r},\sigma)))^{2} &= T_{i}^{2}\\
\frac{2\sigma^{4}}{m_{i}}(ln(\Lagr_{n}(E^{i},\vec{r},\sigma)))^{2} &= m_{i}T_{i}^{2}\\ %shouldn't the 2 be a 4?
\sum_{i}m_{i}T_{i}^{2} &= 2\sigma^{4}\sum_{i}\frac{ln\Lagr_{i}(E^{i},\vec{r},\sigma)^{2}}{m_{i}}
\end{align*}

Let again

\begin{align*}
g(\vec{r})&=\sum_{i}m_{i}T_{i}^{2}
\end{align*}

then

\begin{align*}
\frac{\partial g(\vec{r})}{\partial\sqrt{k}}
 &=2\sum_{i}m_{i}T_{i}\frac{\partial T_{i}}{\partial\sqrt{k}}\\
 &=2\sum_{i}m_{i}T_{i}\frac{\partial(C -\sum_{p\rightarrow k}(E_{ip}-\sqrt{p})^{2})}{\partial\sqrt{k}}\\
 &=-4\sum_{i}m_{i}T_{i}(-\frac{1}{m_{i}})\sum_{fixme}(E_{ik}-\sqrt{k})\\
 &=4\sum_{i}T_{i}\sum_{fixme}(E_{ik}-\sqrt{k})\\
 &=4\sum_{i}T_{i}E_{ik}-4\frac{\sum_{i\rightarrow k}(T_{i}\sqrt{k})}{i\sum_{fixme}T_{i}}\\
\frac{1}{4\sum_{i\rightarrow k}T_{i}}\frac{\partial g(\vec{r})}{\partial\sqrt{k}}
 &=\sum_{i}T_{i}E_{ik}-\sqrt{k}\\
r_{k}^{t+1}&=r_{k}^{t}+\frac{1}{4\sum_{i\rightarrow k}T_{i}}\frac{\partial g(r)}{\partial r_{k}}\\
r^{t+1}&=\vec{r}+\frac{1}{4\sum_{i\rightarrow k}T_{i}}\nabla g\\
\end{align*}

\end{ExampleSection}


\end{Topic}


%\begin{center}
%\begin{figure}[h]
%\begin{center}
%\end{center}
%\caption{} %     title of the Figure
%\label{fig:} % label to refer figure in text
%\end{figure}
%\end{center}

%\newpage

\end{spacing}
\end{document}

